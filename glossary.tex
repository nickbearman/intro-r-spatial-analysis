\documentclass[a4paper,10pt]{article}
\usepackage{geometry}
 \geometry{
 a4paper,
 total={170mm,257mm},
 left=15mm,
 right=15mm,
 top=15mm,
 bottom=10mm,
 }  
\usepackage[utf8]{inputenc}
\usepackage[english]{babel}
\usepackage{graphicx}
\usepackage{wrapfig}
\usepackage{siunitx} %for degree symbol
\usepackage{multicol}
\usepackage[ddmmyyyy,hhmmss]{datetime}
\usepackage{graphicx} % Required for inserting images
\usepackage{datatool} % for pulling info from CSV file
\usepackage{multicol} % for multiple columns

\setlength{\parskip}{0.5em} %gap between paragraphs

\setlength{\columnsep}{5mm} %column separation

\title{R Glossary}
\author{Nick Bearman}
\date{March 2023}

\begin{document}

\maketitle

\section{Introduction}

\begin{multicols}{2}

\textbf{ArcGIS} A commercial GIS software created by ESRI, consisting of \textit{ArcMap}, \textit{ArcCatalog} and \textit{ArcScene} 

\textbf{ArcGIS} A commercial GIS software created by ESRI, consisting of \textit{ArcMap}, \textit{ArcCatalog} and \textit{ArcScene} 

\textbf{ArcMap} Part of \textit{ArcGIS}, the main program for creating and editing spatial data and maps

\textbf{ArcScene} Part of \textit{ArcGIS}, used for 3D data 

\end{multicols}

\section{Introduction 2}

\DTLloadrawdb{list}{glossary-terms.csv} %use DTLloadrawdb rather than DTLloaddb to correctly intrepret $ character. see https://mirror.apps.cam.ac.uk/pub/tex-archive/macros/latex/contrib/datatool/datatool-user.pdf p57 


\DTLforeach{list}{%
\command=Command,\definition=Definition,\example=Example}{\\
\textbf{\command{}} \definition, \texttt{\example{}}.}

\begin{tabular}{|p{4cm}|p{12cm}|} % set width 
% \begin{tabular}{ |c|c| } 

 \hline
 \textbf{Command} & \textbf{Definition} \\ 
 \hline
 cell4 & cell5 \\ 
 \hline
 cell7 & cell8 \\ 
 \hline

\DTLforeach{list}{%
\command=Command,\definition=Definition,\example=Example}{\\
\hline
\texttt{\command{}} & \definition \\ & \texttt{\example{}}} \\
\hline
 
\end{tabular}

New example

\begin{tabular}{|p{4cm}|p{6cm}|p{6cm}|} % set width 
% \begin{tabular}{ |c|c| } 


 \hline
 \textbf{Command} & \textbf{Definition} & \textbf{Exam[ple}\\ 
 \hline
 cell4 & cell5 & cell6 \\ 
 \hline
 cell7 & cell8 & cell9 \\ 
 \hline

 \hline
 \textbf{Command} & \textbf{Definition} & \textbf{Example} \\ 
 \hline
 
\DTLforeach{list}{%
\command=Command,\definition=Definition,\example=Example}{\\
\hline
\textbf{\command{}} & \definition & \texttt{\example{}}} \\
\hline
 
\end{tabular}


\end{document}
