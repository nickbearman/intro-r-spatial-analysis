%line spacing?

%sudo /usr/bin/tlmgr install siunitx to install package
%or install pandoc (download) and then TinyTex in RStudio, and compile in RStudio
%alternativly can compile in Overleaf

\documentclass[a4paper,10pt]{article}
   
\usepackage{geometry}
 \geometry{
 a4paper,
 total={170mm,257mm},
 left=15mm,
 right=15mm,
 top=15mm,
 bottom=10mm,
 }       
\usepackage[utf8]{inputenc}
\usepackage[english]{babel}
\usepackage{graphicx}
\usepackage{wrapfig}
\usepackage{siunitx} %for degree symbol
\usepackage{multicol}
\usepackage[ddmmyyyy,hhmmss]{datetime}
\newcommand{\q}[1]{``#1''} %quotes for "top", from https://tex.stackexchange.com/questions/531/what-is-the-best-way-to-use-quotation-mark-glyphs

\setlength{\parskip}{0.5em} %gap between paragraphs

\setlength{\columnsep}{5mm} %column separation

\begin{document}

\pagenumbering{gobble} %disable page numbering, https://tex.stackexchange.com/questions/7355/how-to-suppress-page-number

\begin{center}

{\huge Glossary: Using R as a GIS}

Terms in \textit{italics} are defined in the glossary, terms with \textbf{brackets()} are R functions

\end{center}

\begin{multicols}{2}

\textbf{\#} Used to precede a comment: \texttt{\#this is a comment}

\textbf{\$} Used to refer to columns within a data frame: \texttt{dataframe\$column} 

\textbf{?} shows the help file for that command \texttt{?help} or \texttt{?head}

\textbf{??} search through the help files for any reference to the word you type: \texttt{??dataframe}

\textbf{\texttt{[,]}} square brackets are used to refer to specific elements in a list or data frame. \texttt{pop2011[1,]} will show the first row and \texttt{pop2011[,1]} will show the first column

\textbf{\{\}} Used in a \texttt{for} loop or a \texttt{if} statement

\textbf{\texttt{<-}} assigns a value or output from a function to a variable

\textbf{abline()} adds vertical lines to a histogram, used to show \textit{classification} breaks: \texttt{abline(v = breaks\$brks, col = "red")}

\textbf{Acknowledgements} Information required on any map, including copyright or data sources  (e.g. for OpenStreetMap or Ordnance Survey Open Data)

\textbf{aggregate()} group the specified data set by a column, applying a function to the values: \texttt{aggregate(x = LSOA\_crimes, by = list(LSOA\_crimes\$lsoa21cd), FUN = length)}

\textbf{BeiDou} Chinese version of \textit{GPS} (see also \textit{GNSS})

\textbf{BNG} (British National Grid) A \textit{projected coordinate system} used to represent locations in Great Britain, consisting of \textit{eastings} and \textit{northings}, e.g. 603125, 112589 (see also \textit{UTM} and \textit{WGS1984}) 

\textbf{c(,)} used to create a list, either numbers \texttt{c(1,2,3)} or \textit{strings} (text) \texttt{c("Thomas","Richard","Harriet")}

\textbf{Categorical} A variable that has a series of values with no inherent order, e.g. country names, also known as \textit{nominal} (see also \textit{variable type}) 

\textbf{Choropleth} A type of mapping where different colours are used to represent difference values; can use \textit{categorical} and \textit{ordinal} data

\textbf{Classes} The groups data are put into for a \textit{choropleth} map

\textbf{Classification} How data are classified into different \textit{classes} for a \textit{choropleth} map (see also \textit{jenks}, \textit{equal count}, \textit{equal interval} and \textit{standard deviation}

\textbf{classIntervals()} calculate the class invervals for the specified data: \texttt{classIntervals(var, n = 6, style = "fisher")}

\textbf{colnames()} shows the names and numbers of the columns in the specified data set: \texttt{colnames(hp.data)}

\textbf{Coordinates} The numbers representing a specific location, usually presented in pairs (see also \textit{latitude}, \textit{longitude}, \textit{WGS1984}, \textit{BNG} and \textit{UTM})

\textbf{CRS} (Coordinate Reference System) The type of coordinates that are used to represent a specific location (see also \textit{WGS1984}, \textit{BNG} and \textit{projection})

\textbf{Correlation} A measure of how much two variables are related, measured using a \textit{R\textsuperscript{2}} value 

\textbf{CSV} (Comma Separated Values) A standard format of \textit{tabular data}, can be opened in Excel  

\textbf{CSVT} An optional file for use with \textit{CSV} files which specifies the \textit{variable type} of each column % in the \textit{CSV} file 

\textbf{Data type} How data is stored within the \textit{Attribute table}, can be \textit{integer} (whole numbers), \textit{real} (decimal numbers) and \textit{string} (text) 

\textbf{DEM} (Digital Elevation Model) a \textit{raster} representation of the height of the earth's surface

\textbf{display.brewer.all()} shows all the potential colour palletts from the \texttt{RColorBrewer} library

\textbf{download.file()} download a file from the specificed URL: \texttt{download.file("http://www.nickbearman.me.uk/
data/r/sthelens.zip","sthelens.zip")}

\textbf{Eastings} A \textit{coordinate} that specifies the distance east, in meters, from the coordinates 0, 0 south-west of the Isles of Scilly (see also \textit{BNG} and \textit{northings})

\textbf{EPSG code} A 4 or 5 digit code used by GIS to define what \textit{CRS} (Coordinate Reference System) a data set is stored in, for example \textit{WGS1984} is \texttt{4326} and \textit{BNG} is \texttt{27700})

\textbf{Equal count} (Quantile) \textit{Classification} method where data are split into a number of groups by putting the same number of data items into each group, also known as \textit{quantile}, see also \textit{classification}

\textbf{Equal interval} \textit{Classification} method where data are split into \textit{classes} that are evenly distributed, e.g. 0-20\%, 20-40\%, etc., see also \textit{classification} 

\textbf{F / FALSE} used to specify we don't want something, \texttt{frame = F} means we don't want a frame

\textbf{Field calculator} Used to calculate new values (e.g. differences) from existing values for all rows in a vector layer, accessed from the \textit{Attribute table} 

\textbf{Fisher} \textit{Classification} method very similar to \textit{Jenks} 

\textbf{for () \{\}} begins a loop to make R repeat a command a set number of times: \texttt {for (i in 1:length(mapvariables))}

\textbf{Galileo} European Union version of \textit{GPS} (see also \textit{GNSS})

\textbf{Geographic coordinate system} A coordinate system covering the whole world, usually using degrees (see also \textit{WGS1984}, \textit{latitude}, \textit{longitude})

\textbf{Geographical Information Science} (GIS) The development of the tools, software and processes used in \textit{Geographical Information Systems} 

\textbf{Geographical Information Systems} (GIS) Using spatial data to answer questions about our world (see also \textit{Geographical Information Science})

\textbf{GeoJSON} Vector spatial data file, consisting of \textit{points}, \textit{lines} and \textit{polygons}; all saved in one file

\textbf{Geopackage} (GPKG) Open format for storing geospatial data, can store multiple \textit{layers} within one file, both \textit{vector} and \textit{raster}, sometimes seen as a replacement for \textit{shapefiles}

\textbf{GLONASS} Russian version of \textit{GPS} (see also \textit{GNSS})

\textbf{GNSS} (Global Navigation Satellite System) formal generic term for satellite location systems, see also \textit{GPS}, \textit{GLONASS}, \textit{Galileo} and \textit{BeiDou}

\textbf{GPS} (Global Positioning System) a series of 24 satellites in orbit around the earth which allow a GPS device to locate itself, with an accuracy of 1m to 10m  (see also \textit{GNSS}, \textit{GLONASS}, \textit{Galileo} and \textit{BeiDou})

\textbf{group\_by()} used to group data by a specific value

\textbf{head()} used to show the first six rows of the data frame: \texttt{head(hp.data)}

\textbf{hist()} shows a histogram of the specified data: \texttt{hist(var)}

\textbf{if () \{\} } begins an if loop to make R choose a specific command based on a cirteria

\textbf{Inset Map} A small map included on the main map to aid orientation, e.g. a map of Ghana might include an \textit{inset map} of Africa to show where Ghana is

\textbf{install.packages()} Install (download from the internet) new \textit{libraries} in R, you only need to do this once, remember to include the quotes \texttt{install.packages("sf")}

\textbf{Integer} A whole number used to represent data, can be used in a \textit{choropleth} map (see also \textit{data type}) 

\textbf{Jenks} (natural breaks) \textit{Classification} method based on the Jenks algorithm which groups similar data values together, also known as \textit{natural breaks}, see also \textit{classification}

\textbf{Joining} The process of linking attribute information to spatial data, often used so the information can be shown on a \textit{choropleth} map 

\textbf{Latitude} A \textit{coordinate} that specifies the distance north or south, ranging from \ang{0} at the Equator to \ang{90} (North or South) at the poles (see also \textit{WGS1984} and \textit {longitude})

\textbf{Layers} When you add data into a GIS each different file appears as a different \textit{layer}; this allows different datasets to be overlaid on one another (see also \textit{Contents})

\textbf{Layout} The term for where you create your final map, where you can add maps, \textit{legends}, \textit{scale bars}, \textit{north arrows}, etc. (also known as \textit{Print Layout}, for QGIS)

\textbf{Legend} An important part of any map, showing what the symbols or colours used on the map represent 

\textbf{library()} load the specified library: \texttt{library(sf)} you need to do this everytime you start \textit{RStudio}

\textbf{Libraries} (or packages) A collection of R functions brought together to extend R's functionality, we use the \texttt{sf} and \texttt{tmap} libraries, installed with \texttt{install.packages("")} and loaded with \texttt{library()}

\textbf{Lines} Used in \textit{vector} data sets to indicate a linear feature, such as rivers, roads or railways; is a series of \textit{points} joined together in a certain order

\textbf{Longitude} A \textit{coordinate} that specifies the distance east or west, ranging from \ang{0} at the Prime Meridian (Greenwich, London, UK) to \ang{180} around the international data line (see also \textit{WGS1984} and \textit{latitude})

\textbf{MapInfo} A commercial GIS software, created by Pitney Bowes, now developed by Precisely 

\textbf{mean()} calculate the mean of the specified set of values: \texttt{mean(house.prices)}

\textbf{merge()} joins two data frames together using a specified attribute or ID: \texttt{merge(LSOA, pop2021, by.x="lsoa21cd", by.y="geography.code")}

\textbf{Natural breaks} (Jenks / Fisher) \textit{Classification} method based on the \textit{Jenks} algorithm which groups similar data values together, see also \textit{classification}

\textbf{Nominal} A variable that has a series of values with no inherent order, e.g. country names, also known as \textit{categorical} (see also \textit{variable type}, \textit{ordinal} and \textit{quantitative})

\textbf{North arrow} Used to show the direction of North on a map, used to aid orientation (see also \textit{inset map})

\textbf{Northings} A coordinate that specifies the distance north, in meters, from the \textit{coordinates} 0, 0 south-west of the Isles of Scilly (see also \textit{BNG} and \textit{eastings})

\textbf{Ordinal} Similar to a categorical variable, but with a clear order, e.g. high priority, medium priority, and low priority (see also \textit{variable type}, \textit{quantitative}) 

\textbf{Pixel} An individual unit in a \textit{raster} data set, the size of the \textit{resolution} squared (i.e. for a 100m resolution \textit{raster} data set, each \textit{pixel} would be 100m x 100m, covering 10,000 square meters (or 1 hectare) of land)

\textbf{plot()} Base function in R used for plotting graphs and sometimes raster data

\textbf{Points} A \textit{vector} data type used to indicate a specific location, such as sample collection points, bird nest sites, towns or cities

\textbf{Polygons} A \textit{vector} data type used to indicate areas, e.g. land parcels, counties and fields; is a series of \textit{points} joined in a certain order with the last point linked back to the first point to indicate an area

\textbf{print()} Command used to output some information to the console

\textbf{Print Layout} The term for where you create your final map (see also as \textit{Layout})

\textbf{project file} A file created by GIS to store the symbology, classification, layouts and a list of the data layers you are using. Project files do not store any spatial data. Examples include .qgz / .qgs (QGIS) and .arpx (ArcGIS Pro)

\textbf{Projected coordinate system} A coordinate system covering a specific, small area, usually uses meters and can me used for measuring distances (see also \textit{BNG}, \textit{easting}, \textit{northing} and \textit{UTM})

\textbf{Projection} The way the sphere shaped earth is distorted to fit on a flat piece of paper (see also \textit{WGS1984}, \textit{BNG} and \textit{coordinate system})

\textbf{QGIS} An open source GIS free for anyone to download, use and improve

\textbf{QGIS project file} (.qgz / .qgs) (QGIS) A project file for \textit{QGIS} which contains links to all the data files (such as \textit{shapefiles} and/or \textit{GeoJSON} files) and information on how they are symbolised; the \textit{project file} does not contain the data itself

\textbf{qtm()} a quick way of showing spatial data, either just the data itself, \texttt{qtm(LSOA)} or a quick choropleth map \texttt{qtm(LSOA\_crimes\_aggregated, fill = "count of crimes")}

\textbf{Quantile} (equal count) \textit{Classification} method where data are split into a number of groups by putting the same number of data items into each group, also known as \textit{equal count}, see also \textit{classification}

\textbf{Quantitative} A numeric variable with an inherent order, e.g. GDP per capita, (see also \textit{variable type})

\textbf{R} Scripting language designed initially for statistics, expanded with \textit{libraries} to include spatial analysis, usually used through  \textit{RStudio}

\textbf{R\textsuperscript{2}} The \textit{correlation} coefficient of two different data sets, a value of 1 is a strong positive \textit{correlation}, -1 is a strong negative \textit{correlation}

\textbf{Raster} A type of spatial data used with GIS, consisting of a regular grid of points spaced at a set distance (the \textit{resolution}); often used to represent heights (DEM) or temperature data (see also \textit{vector})

\textbf{Raster calculator} Used with \textit{raster} data to calculate differences (subtract) or calculate other indices (e.g. NDVI)

\textbf{read.csv()} read in a \textit{csv} file from the specified path: \texttt{read.csv("police-uk-2020-04-merseyside.csv")}

\textbf{Real} A decimal number used to represent data, can be used in a \textit{choropleth} map (see also \textit{data type}) 

\textbf{res()} command to display the \textit{resolution} of a \textit{raster} data set

\textbf{Resolution} The size of each \textit{pixel} in a \textit{raster} data set (e.g. 100 meters, 1km, 100km)

\textbf{RStudio} easy to use interface for \textit{R}, allows easy access to \textit{variables}, plot window and use of \textit{scripts}, known as an IDE (integrated development environment)

\textbf{Sat-nav} A navigation system in cars, which uses \textit{GPS} to direct the driver to their destination

\textbf{Scale} The ratio of units of distance on the map to units of distance in the real world; for example 1:25,000 means that 1cm on the map represents 25,000cm (or 250m) in the real world; usually shown on a \textit{scale bar}

\textbf{Scale bar} Used to show the \textit{scale} of a map

\textbf{Scripts} a series of \textit{R} commands that can be run either individually or all together at once

\textbf{setwd()} set the working directory to the specified folder: \texttt{setwd("C:/Users/nick/Documents/GIS")}

\textbf{Shapefile} A type vector of spatial data file, consisting of one of \textit{points}, \textit{lines} or \textit{polygons}; represented in \textit{GIS} as one file but in fact consisting of multiple files (between 4 and 6 files, with extensions of .shp, .shx, .dbf and .prj)

\textbf{spatial autocorrelation} A measure of how much the location of spatial data is correlated with itself, a value of 1 is a strong positive autocorrelation meaning data are spatially clustered with low values near each other and high values near each other, 0 is no spatial autocorrelation meaning data are distirbuted at random and space is not important in this data set, -1 is a strong negative correlation with low and high values as mixed as possible

\textbf{st\_as\_sf()} use to create a spatial data frame from a non spatial data frame: \texttt{st\_as\_sf(crimes, coords = c('Longitude', 'Latitude'), crs = 4326)}

\textbf{st\_join()} spatial join, joining data by location: \texttt{st\_join(LSOA, crimes\_sf\_bng)}

\textbf{st\_read()} read in a shapefile or other spatial data file: \texttt{st\_read("england\_lsoa\_2021.shp")}

\textbf{st\_transform()} reproject the specified spatial data object to the specified \textit{crs}: \texttt{st\_transform(crimes\_sf, crs = 27700)}

\textbf{st\_write()} save the specified spatial data object, e.g. as a shapefile:  \texttt{st\_write(LSOA\_crimes\_aggregated, "LSOA-crime-count.shp")}

\textbf{Standard deviation} \textit{Classification} method based on standard deviation and mean of the data set  

\textbf{String} A piece of text (e.g. a name) used to represent data, cannot be used to create a \textit{choropleth} map (see also \textit{data type}, \textit{real} and \textit{integer} 

\textbf{Style} (QGIS) / \textbf{Symbology} (ArcGIS Pro) The options to choose the colours and/or symbols to represent data on the map; accessed through right-clicking on the layer and selecting properties and navigating to the Style tab)

\textbf{style} (R) Parameter used to specify which classification method to use for a choropleth map: \texttt{style = "equal"}

\textbf{T / TRUE} used to specify that we do want something: \texttt{legend.hist = T} means we want to show the legend histogram

\textbf{table()} count how many different values there are in the specificed column, and output as a table: \texttt{table(crimes\_sf\_bng\$Crime.type)}

\textbf{Tabular data} Data laid out in rows and columns, as used in Excel (see also \textit{CSV})

\textbf{title.position} specify the location of the title on a map: \textit{title.position = c(0.7, \q{top})}

\textbf{tm\_borders()} specify the colour and thickness of a border around a map

\textbf{tm\_compass()} specify the location of the north arrow or compass on a map: \texttt{tm\_compass(position = c(0.3, 0.07))}

\textbf{tm\_dots()} specify the symbology of a point layer on a map: \texttt{tm\_dots(size = 0.1, shape = 19, col = "darkred", alpha = 0.5)}

\textbf{tm\_layout()} specify a range of options in the map layout: \texttt{tm\_layout(frame = F, title = "Liverpool", title.size = 2)}

\textbf{tm\_lines()} specify the symbology and options of a line layer on a map: \texttt{tm\_lines(col = "black")}

\textbf{tm\_polygons()} specify the symbology and options of a polygon layer on a map: \texttt{tm\_polygons("Age00to04", title = "Aged 0 to 4", palette = "Greens", n = 6, style = "jenks")}

\textbf{tm\_raster()} specify and plot a raster data set: \texttt{tm\_raster(palette = "Greens")}

\textbf{tm\_scale\_bar()} specify the location and style of the scale bar on a map \texttt{tm\_scale\_bar(width = 0.22, position = c(0.05, 0.18))}

\textbf{tm\_shape()} specify the layer shown on a map, often used in conjunction with \textit{tm\_polygons()}: \texttt{tm\_shape(LSOA)}

\textbf{tmap\_mode()} change the tmap mode from plot (default) to view (slippy map with basemap) or back: \texttt{tmap\_mode("view"),  tmap\_mode("plot")}

\textbf{tmap\_save()} save the map to a specified output file: \texttt{tmap\_save(m, filename = paste0("map-",mapvariables[i],".png"))}

\textbf{unzip()} unzip a file: \texttt{unzip("sthelens.zip")}

\textbf{UTM} (Universal Transverse Mercator) A type of \textit{projected coordinate system} used to represent any location in the world, consisting of a series of zones and a set of \textit{coordinates} for each zone, in meters (see also \textit{BNG} and \textit{WGS1984}) 

\textbf{Variable type} Information on the type of information within a variable, can be \textit{categorical}, \textit{ordinal} or \textit{nominal} 

\textbf{Vector} A type of spatial data used with \textit{GIS}, consisting of \textit{points}, \textit{lines} and \textit{polygons} (see also \textit{raster})

\textbf{Vertex (vertices)} Name for each of the points that connect the \textit{line} segments of a \textit{line} or \textit{polygon} \textit{shapefile}

\textbf{View()} view the specified data in a new tab in RStudio: \texttt{View(hp.data)}

\textbf{WGS1984} A \textit{coordinate system} used to represent any location in the world, consisting of \textit{latitude} and \textit{longitude} e.g. 51.0426 N, 1.3772 E or \ang{51} 2’ 33.53’’ N, \ang{1} 22’ 38.23’’ E (see also \textit{BNG} and \textit{UTM}) 

\textbf{which()} function to select which elements or rows match the specified criteria: \texttt{which(tram\_stations\$RSTNAM == "Chorlton")}

\end{multicols}

\begin{center}

{\footnotesize \textit{This glossary was last updated on {\today} by Dr. Nick Bearman (nick@nickbearman.com) and is written in LaTeX. This work is licensed under the Creative Commons Attribution-ShareAlike 4.0 International License, http://creativecommons.org/ licenses/by-sa/4.0/deed.en. The latest version of the PDF is at https://github.com/nickbearman/intro-r-spatial-analysis.}}

\end{center}

\end{document}